\documentclass[11pt, a4paper]{article}
\title{Wacky Racers 2019 Game Board Documentation}
\author{Michael Hayes, Ben Mitchell, Matthew Edwards}
\date{Version 1, \today}

\usepackage[margin=1in]{geometry}
\usepackage{parskip}
\usepackage{float}
\usepackage{todonotes}
\presetkeys{todonotes}{inline}{}
\usepackage{graphicx}
\usepackage[T1]{fontenc}
\usepackage{makecell}
\usepackage{hyperref}
\usepackage{tabularx}


\begin{document}
\maketitle

\section{Executive Summary}
You should ensure that the status LED is easily visible and identifiable on both your racer and your hat.
You should implement the commands in the order they are given in Table~\ref{tab:board-commands}, and U, M, N, B, K and R are required for the competition.


\section{Introduction}
Each sub-group must assemble a game board as part of the SMT Lab induction process.

The game board interfaces to each Wacky Racer and Wacky Hat using a UART interface.
Its purpose is to communicate with a game server over WiFi for coordination of competitions.
It controls a string of programmable LEDs used to identify each Wacky Hat and Wacky Racer.
It also interfaces to a bump sensor that is required for entry into the capture the flag competition.

\section{Functionality}
The first LED on the LED string (that is, the LED closest to the connector) is used as a status indicator.
Depending on the game board's state, the status LED either indicates which mode the board is in or which error has occurred.
\textbf{You should ensure that the status LED is easily visible and identifiable on both your racer and your hat.}

The game board has three modes of operation: demo mode, test mode and competition mode.
The default mode is demo mode; this is the mode the board will enter if you do not respond (or do not respond correctly) to the startup commands.
In demo mode, the status LED is white and the rest of the LEDs display a rainbow pattern.

Your response to the M command determines whether the board enters test mode or competition mode.
Outside of events organised by Michael or the TAs, you should use test mode.
For your convenience, it is recommended to use a switch or jumper on your board to control this.
In test mode, the status LED is breathing blue and the game board hosts a wireless network named after your group number and board type (e.g.,~``Wacky Racers Group 1 Racer'').
On this network, a web interface is available at \url{http://wacky.test/} which allows you to test various aspects of your board's interaction with the game board.

In competition mode, the status LED is breathing green and the game board connects to the competition wireless network.
Your game board will be controlled remotely by the game server, allowing you to compete in the event.

The other status LED colours are given in Table~\ref{tab:status-led}.

\begin{table}[H]
  \centering
  \begin{tabular}{l l}
    White  & Demo mode (or briefly, requesting board information).  \\
    Red    & Could not connect to competition network.  \\
    Orange & Connecting to competition network.  \\
    Yellow & Connecting to game server.  \\
    Green, breathing & Competition mode.  \\
    Blue, breathing & Test mode.  \\
    Red, flashing quickly & Something else went wrong.  \\
    Purple & OTA update (or update check) in progress.
  \end{tabular}
  \caption{Status LED colours.}
  \label{tab:status-led}
\end{table}


\section{Serial Interface}
The game board send commands over a UART interface and expects responses.
The responses must begin within 30\,ms.
The software interface is shown in Table~\ref{tab:board-commands}.
The baud rate is 250000 and 8 bits are sent with no parity and 1 stop bit.
The maximum response length is 100 bytes, except for the LED pattern which has a maximum response length of 280 bytes (40 colours, 39 commas and a newline).

\textbf{You should implement the commands in the order they are given in Table~\ref{tab:board-commands}.}
Not responding to the U, M, N and B commands will put the game board into demo mode.
K and R are also required for the competition.
L allows you to customize your LED strip displays at certain times; it will be sent at regular intervals so you can implement animations.
T is only necessary if your radio link does not work, but it does allow you to control your racer from a laptop or mobile device (using the test mode interface).
If you receive a command for which you have not implemented a response, you must discard up to the next newline character.


\begin{table}[H]
  \centering
  \begin{tabularx}{\textwidth}{l X}
    U & Return how much current (in mA) you can provide on the 5V rail (this should be lower when powered over USB).  \\
    M & Return `comp' for competition mode and `test' for test mode. \\
    N & Return group number. \\
    B & Return board type (`hat' or `racer'). \\
    K & For racer, kill motors, prepare to die!  For hat, play sound.  Return an empty string. \\
    R & Re-enable motors.  Return an empty string. \\
    L\texttt{x} & Return LED pattern to display on LED strip, where \texttt{x} is 0 in test mode, 1 for victory, and 2 for loss.  The LED pattern should be a `<' character, followed by 40 colour values separated by commas (where each colour value is a 6-digit RRGGBB hexadecimal number), then a `>' character.  If fewer than 40 colours are sent, the pattern will be repeated along the strip.  \\
    G & Return group name. \\
    S & Return student names (comma separated list). \\
    V & Return battery voltage in mV as an integer.  \\
    I & Return battery current in mA as an integer (racer only). \\
    Q & Return battery charge used in mAh formatted as an integer (racer only). \\
    T & For hat, return speed command as
    `\texttt{linear},\texttt{angular}' where each component is an
    integer percentage between -100 and 100 inclusive.
    Return an empty string to use your own radio link.  \\
    T\texttt{l},\texttt{a} & For racer, move with forward speed \texttt{l} and rotation rate \texttt{a} where the speeds are integer percentages.  Return an empty string. \\
    Z & Return an empty string and enter low-power mode. \\
  \end{tabularx}
  \caption{Game board commands.  All commands and responses are
    newline-terminated ASCII strings.}
  \label{tab:board-commands}
\end{table}


\end{document}
